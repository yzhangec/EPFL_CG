We are using a couple of C++11 features, so please ensure that you use an up-\/to-\/date compiler (G\+CC 4.\+7 or higher, Visual Studio 2012). \href{www.cmake.org}{\tt C\+Make} is used for setting up build environments.

\subsection*{Building under Linux/mac\+OS }

Inside the exercise\textquotesingle{}s top-\/level directory, execute the following commands\+: \begin{DoxyVerb}mkdir build
cd build
cmake ..
make
\end{DoxyVerb}


The last command -- i.\+e. {\ttfamily make} -- compiles the application. Rerun it whenever you have added/changed code in order to recompile.

To build a pretty documentation use\+: \begin{DoxyVerb}make doc
\end{DoxyVerb}


and open the {\ttfamily index.\+html} in the html folder with your favourite browser. To build the documentation, you must install Doxygen.

\subsection*{Building with X\+Code (mac\+OS) }

If you wish, you can use the C\+Make build system to generate an X\+Code project. Inside the exercise\textquotesingle{}s top-\/level directory, execute the following commands\+: \begin{DoxyVerb}mkdir xcode
cd xcode
cmake -G Xcode ..
open RayTracing.xcodeproj
\end{DoxyVerb}


\subsection*{Building under Microsoft Windows (Visual Studio) }


\begin{DoxyItemize}
\item Start the C\+Make-\/\+G\+UI.
\item Open the Ray\+Tracing top-\/level directory as source directory.
\item Choose a build directory.
\item Click on configure and select Visual Studio as generator.
\item Click generate to create the Visual Studio project files.
\item Open the Visual Studio solution file that is in the build directory you chose in C\+Make.
\end{DoxyItemize}

\subsection*{Running the \hyperlink{classRay}{Ray} Tracer }

The program expects two command line arguments\+: an input scene ({\ttfamily $\ast$.sce}) and an output image ({\ttfamily $\ast$.tga}). To render the scene with the three spheres, while inside the {\ttfamily build} directory, type\+: \begin{DoxyVerb}./raytrace ../scenes/spheres/spheres.sce output.tga
\end{DoxyVerb}


If you have finished all exercise tasks, use \begin{DoxyVerb}./raytrace 0
\end{DoxyVerb}


to render all scenes at once.

To set the command line parameters in M\+S\+VC or Xcode, please refer to the documentation of these programs (or use the command line...).

\subsection*{Assignment 3\+: Triangle Meshes }

In this assignment you will need to edit the \hyperlink{Mesh_8cpp}{Mesh.\+cpp} file.


\begin{DoxyItemize}
\item Compute vertex normals weighted by opening angles in \hyperlink{classMesh_a4e9bedfc415b7135c9587f63535fcb6d}{Mesh\+::compute\+\_\+normals()}.
\item Compute the ray-\/triangle intersection with barycentric coordinates using Cramer\textquotesingle{}s rule within \hyperlink{classMesh_a9be7264791ff3de7dbf99f8548fb7725}{Mesh\+::intersect\+\_\+triangle()} function. For intersections normals use triangle normals when flat shading or interpolate vertex normals when Phong shading.
\item To improve the computation time use the axis-\/aligned bounding box test for triangle meshes. Implement the ray-\/box intersection within \hyperlink{classMesh_a5839749bb09a6bf09c56056016cc11b2}{Mesh\+::intersect\+\_\+bounding\+\_\+box()}.
\item (not graded) To parallelize your computation, install T\+BB and make sure you have the \hyperlink{classScene_aeaecd6069dfc02986fd04c9a8f905e89}{Scene\+::render()} function provided with this assignment.
\end{DoxyItemize}

For more details, please refer to the assignment handout and lecture+exercise slides. 